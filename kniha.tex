\documentclass[a4paper,8pt]{book}
\usepackage[utf8]{inputenc}
\usepackage[czech]{babel}

\title{ImproWiki}
\author{wiki.improliga.cz pod licencí Creative Commons Uveďte autora}
\makeatletter
\renewcommand\thesection{}
\renewcommand\thesubsection{}
\makeatother
\usepackage[margin=1.6cm]{geometry}
\usepackage{tgpagella}
\usepackage[T1]{fontenc}
\newcommand{\todo}[1]{
Ve wiki či v převodníku se píše že by se mělo udělat: #1 
}
\newcommand{\katabox}[3]{
 \begin{tabular}[t]{r|l}
\textbf{Téma} & #1 \\ 
\textbf{Hráči} & #2 \\
\textbf{Čas} & #3 \\
\end{tabular}
}

\newcommand{\odkaz}[2]{
\textbf{#1}
%\footnote{\textbf{#2}, strana \pageref{#2} }
\textsuperscript{s.~\pageref{#2}}
} 



\begin{document}
\begin{titlepage}
\maketitle
\end{titlepage}
 
\chapter{Úvod}\label{úvod}


Oproti verzi dostupné na internetu zde chybí odkazy na videa a další internetové zdroje.

Na textech se podíleli Václav Černý, Pavel Žák, Vanda Gabrielová, Vojtěch Kopta a Jan Drahorád

Aktuálnost a užitečnost tištěné verze není zaručena, v případě nejasností konzultujte se zkušenějším hráčem či s improwiki.

\LaTeX{} konverzi připravil  Václav Černý. 
\chapter{Zápas}
%\label{zápas}
\input{zapas}

\chapter{Postavy}
%\label{zápas}
\input{postavy}


\chapter{Zápasové kategorie}\label{zápasové kategorie}
\textbf{Varování:} Popis kategorí je určen spíše zkušeným hráčům pro připomenutí, než nováčkům na naučení se.  
\input{kategoriez}

\chapter{Další kategorie}\label{další kategorie}
Tyto kategorie bývají hrány na improshow.
\input{kategoriei}

\chapter{Rozcvičky}\label{rozcvičky}
\input{rozcvicky}

\chapter{Cvičení}\label{cvičení}
\input{cviceni}

\chapter{Fauly}\label{fauly}
\input{fauly}

\chapter{Terminologie}\label{terminologie}
\input{terminologie}


\chapter{Příběh}\label{pribeh}
\input{pribeh}

\chapter{Knihy}\label{knihy}
\input{books}


\chapter{Autoři}\label{autori}
\section{Jan Drahorád}
Zatím nendodal svůj medailonek.

\input{authors}

\chapter{Co se jinam nevešlo}\label{co se jinam nevešlo}
\input{zbytek}
\chapter{Todo}
\todo{infobox kategorie}
\todo{Hlasovací arch}
\todo{br}
\todo{živé rekvizity - na webu}
\todo{vysílání fm - odkaz}
\todo{jména uživatelů}
\setcounter{tocdepth}{1}
\tableofcontents
\end{document}

\documentclass[a4paper,10pt,openany]{book}
\usepackage[utf8]{inputenc}
\usepackage[czech]{babel}
\usepackage{graphicx}
\usepackage{wrapfig}
\usepackage{longtable}
\title{ImproWiki}
\author{wiki.improliga.cz pod licencí Creative Commons Uveďte autora}
\makeatletter
\renewcommand\thesection{}
\renewcommand\thesubsection{}
\makeatother
\usepackage[margin=1.6cm]{geometry}
\usepackage{tgpagella}
\usepackage{multicol}

\usepackage[T1]{fontenc}
\newcommand{\todo}[1]{

Ve wiki či v převodníku se píše že by se mělo udělat: \textbf{#1} 

}
\newcommand{\katabox}[3]{\begin{wrapfigure}{R}{5.3cm}
 \begin{tabular}[t]{r|p{4cm}}
\textbf{Téma}&\p{ #1} \\ 
\textbf{Hráči}&\p{ #2} \\
\textbf{Čas}&\p{#3} \\
\end{tabular}
\end{wrapfigure}}

\newcommand{\obrazek}[2]{
\begin{wrapfigure}{r}{5cm}
\begin{center}
\includegraphics[width=3cm]{#1}
\end{center}
#2
\end{wrapfigure}


}

\newcommand{\odkaz}[2]{\textbf{#1}\textsuperscript{s.~\pageref{#2}}} 

%\footnote{\textbf{#2}, strana \pageref{#2} }

\clubpenalty 10000
\widowpenalty 10000
\begin{document}
\begin{titlepage}
\maketitle
\end{titlepage}
 
\chapter{Úvod}\label{úvod}

\input{uvod} 

Oproti verzi dostupné na internetu zde chybí odkazy na videa a další internetové zdroje.

Na textech se podíleli \odkaz{Václav Černý}{uživatel:vatoz}, \odkaz{Pavel Žák}{uživatel:just-paja}, \odkaz{Vanda Gabrielová}{uživatel:vandagabi}, \odkaz{Vojtěch Kopta}{uživatel:vojtechkopta} a Jan Drahorád

Aktuálnost a užitečnost tištěné verze není zaručena, v případě nejasností konzultujte se zkušenějším hráčem či s improwiki.

Tučně vysázená \odkaz{hesla}{úvod} říkají, na které straně se o daném termínu dočtete víc.

Některé termíny ještě nejsou popsané, přihlašte se na http://wiki.improliga.cz/ a dopište je.

\textbf{Varování:} Nikdo se ještě nenaučil improvizovat z knížky. Popis kategorí je určen spíše zkušeným hráčům pro připomenutí, než nováčkům na naučení se.  

\LaTeX{} konverzi připravil  Václav Černý. 
\chapter{Zápas}
%\label{zápas}
\input{zapas}






\chapter{Postavy}
%\label{zápas}
\input{postavy}


\chapter{Zápasové kategorie}
\label{zápasové kategorie}
\label{:kategorie:zápasové kategorie}
\input{kategorie_start}
\input{kategoriez}

\chapter{Kategorie na improshow}\label{další kategorie}
\label{:kategorie:kategorie na improshow}
Tyto kategorie bývají hrány na improshow.
\input{kategoriei}

\chapter{Rozcvičky}\label{rozcvičky}
\input{rozcvicky_start}
\input{rozcvicky}

\chapter{Rozcvičky}\label{rozcvičky}
\input{rozcvicky_start}
\input{rozcvicky}



\chapter{Předzápasový trénink}\label{předzápasový trénink}
\input{predzapasovy}

\chapter{Fauly}\label{fauly}
\input{fauly_start}
\input{fauly}

\chapter{Terminologie}\label{terminologie}
\label{:kategorie:terminologie}
\input{terminologie}

\chapter{Příběh}\label{příběh}
\input{pribeh_start}
\input{pribeh}

\chapter{Knihy}\label{knihy}
\input{books}


\chapter{Autoři}\label{autori}
\section{Jan Drahorád}
Zatím nedodal svůj medailonek.

\input{authors}

\chapter{Co se jinam nevešlo}\label{co se jinam nevešlo}
\input{zbytek}
\chapter{Todo}
\todo{infobox kategorie}
\todo{Hlasovací arch}
\todo{br}
\todo{vysílání fm - odkaz}
\todo{jména uživatelů}
\todo{Obrázky}
}

\chapter{Bonusy}
Věci zde jsou generované z wiki.
\todo{tady nechci mít číslování stránek}
\pagebreak
\newcommand{\btbinfo}[6]{

\ifx Z#6 \textbf{#1} \else #1 \small{  (i)} \fi  & #6 & \pageref{#2} &  \small{#3} & \small{#4} & \small{#5}\hline
} 
\begin{longtable}{p{4cm}|p{6mm}|p{.4cm}|p{3.2cm}|p{3cm}|p{4.2cm}|}
Název&Z&str.&čas&hráči&téma \hline 
\input{boxtable}
\end{longtable}
\pagebreak

\renewcommand{\btbinfo}[6]{
\ifx Z#6 \large{#1}   \fi

} 



\begin{multicols}{2}

 \begin{tabular}[t]{p{4cm}||p{4cm}}
\large{Zápas} \\

Rozhodčí:  & Konferenciér: \\
Pomocňáci:    \\
Hudba: & Světla: \\
Tým: &  Tým\\
1. \small{(cpt)}  & 1.\small{(cpt)} \\
2.  & 2. \\
3. & 3. \\
4.  & 4. \\
Body:  & Body: \\


\end{tabular}


\input{boxtable}
\end {multicols}

\renewcommand{\btbinfo}[6]{
\ifx Z#6  \else \large{#1}  \fi

}
\begin{multicols}{2}
\input{boxtable}
\end {multicols}


\setcounter{tocdepth}{1}
\tableofcontents
\end{document}

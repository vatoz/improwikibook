\documentclass[a4paper,10pt,openany]{book}
\usepackage[utf8]{inputenc}
\usepackage[czech]{babel}
\usepackage{graphicx}
\usepackage{wrapfig}
\usepackage{longtable}
\usepackage{needspace}
\usepackage{titlesec}
\title{ImproWiki}
\author{wiki.improliga.cz pod licencí Creative Commons Uveďte autora}
\makeatletter
\renewcommand\thesection{}
\renewcommand\thesubsection{}
\makeatother
\usepackage[margin=1.6cm]{geometry}
\usepackage{tgpagella}
\usepackage{multicol}

\usepackage[T1]{fontenc}
\newcommand{\todo}[1]{

TODO: \textbf{#1} 

}

\setlength{\intextsep}{0pt}


\newcommand{\katabox}[3]{\begin{wrapfigure}{R}{5.3cm}
 \begin{tabular}[t]{r|p{4cm}}
\textbf{Téma}&\p{ #1} \\ 
\textbf{Hráči}&\p{ #2} \\
\textbf{Čas}&\p{#3} \\
\end{tabular}
\end{wrapfigure}}

\newcommand{\faulbox}[3]{

\begin{wrapfigure}{L}{0}
#1
\end{wrapfigure}

\textbf{Gesto:}\p{#2} 


\textbf{Trestné body:}\p{#3} 


}


\newcommand{\obrazek}[2]{
\includegraphics[width=4cm]{#1}

\textbf{#2}}


\newcommand{\obrazekmaly}[1]{
\includegraphics[width=1.8cm]{#1}}


\newcommand{\odkaz}[2]{\textbf{#1}\textsuperscript{s.~\pageref{#2}}} 

%\footnote{\textbf{#2}, strana \pageref{#2} }

\clubpenalty 10000
\widowpenalty 10000


\titlespacing*{\section}{-5mm}{4mm}{3mm}

\begin{document}
\hyphenation{pros-tře-dí-mi dvo-ji-ce-mi}
\begin{titlepage}
\begin{center}

% Upper part of the page. The '~' is needed because \\
% only works if a paragraph has started.
\includegraphics[width=0.15\textwidth]{./logo}~\\[2cm]

\textsc{\Large wiki.improliga.cz pod licencí Creative Commons Uveďte autora }\\[0.5cm]

% Title
\HRule \\[0.4cm]
{ \huge \bfseries ImproWiki \\[0.4cm] }

\HRule \\[1.5cm]


\vfill

% Bottom of the page
{\large Vygenerováno \today}

\end{center}




\end{titlepage}
 
\chapter{Úvod}\label{úvod}

\input{uvod} 

Oproti verzi dostupné na internetu zde chybí odkazy na videa a další internetové zdroje.

Aktuálnost a užitečnost tištěné verze (ani wiki) není zaručena, v případě nejasností konzultujte se zkušenějším hráčem či s improwiki.

Tučně vysázená \odkaz{hesla}{úvod} říkají, na které straně se o daném termínu dočtete víc.

Některé termíny ještě nejsou popsané, přihlašte se na http://wiki.improliga.cz/ a dopište je.

\textbf{Varování:} Nikdo se ještě nenaučil improvizovat z knížky. Popis kategorí je určen spíše zkušeným hráčům pro připomenutí, než nováčkům na naučení se.  

\LaTeX{} sazbu připravuje  \odkaz{Václav Černý}{uživatel:vatoz}, chyby či náměty na vylepšení konverze hlaste na https://github.com/vatoz/improwikibook/issues

Případné faktické chyby a překlepy můžete opravit přímo na improwiki.
\chapter{Zápas}
\input{zapas}






\chapter{Postavy}
\input{postavy}


\chapter{Zápasové kategorie}
\label{zápasové kategorie}
\label{:kategorie:zápasové kategorie}
\input{kategorie_start}
\input{kategoriez}

\chapter{Kategorie na improshow}\label{další kategorie}
\label{:kategorie:kategorie na improshow}
Tyto kategorie bývají hrány na improshow.
\input{kategoriei}

\chapter{Rozcvičky}\label{rozcvičky}
\label{:kategorie:rozcvičky}

\input{rozcvicky_start}
\input{rozcvicky}


\chapter{Připrava zápasu}
\input{priprava}

\chapter{Předzápasový trénink}
\input{predzapasovy}

\chapter{Cvičení}
\input{cviceni}

\chapter{Fauly}\label{fauly}

\label{:kategorie:fauly}
\input{fauly_start} 
\input{fauly} 

\chapter{Terminologie}\label{terminologie}
\label{:kategorie:terminologie}
\input{terminologie}

\chapter{Příběh}
\input{pribeh_start}
\input{pribeh}

\chapter{Knihy}\label{knihy}
\input{books}


\chapter{Autoři}\label{autori}
\section{Jan Drahorád}
Zatím nedodal svůj medailonek, často je na zápasech jako \odkaz{hlavní rozhodčí}{rozhodčí} nebo neukázněný divák.

\input{authors}

\chapter{Co se jinam nevešlo}\label{co se jinam nevešlo}
\input{zbytek}

\chapter{Bonusy}
\pagebreak
\newcommand{\btbinfo}[6]{

\ifx Z#6 \textbf{#1} \else #1 \small{  (i)} \fi   & \pageref{#2} &  \small{#3} & \small{#4} & \small{#5}\hline
}
\thispagestyle{empty} 
\begin{longtable}{|p{4cm}|p{.4cm}|p{3.2cm}|p{3cm}|p{4.2cm}|}
\hline 
Název&str.&čas&hráči&téma \hline 
\input{boxtable}

\end{longtable}
\thispagestyle{empty} 
\pagebreak
\thispagestyle{empty} 
\renewcommand{\btbinfo}[6]{
\ifx Z#6 \large{#1}   \fi

} 



\begin{multicols}{2}

 \begin{tabular}[t]{|p{4cm}|p{4cm}    |}
 \hline 
 \multicolumn{2}{|c|}{\large{Zápas}} \\
  \hline
Rozhodčí:  & Konferenciér: \\
 \hline
\multicolumn{2}{|l|}{ Pomocňáci: }   \\ \hline
Hudba: & Světla: \\ \hline \hline
Tým: &  Tým:\\ \hline
1. \small{(cpt)}  & 1.\small{(cpt)} \\ \hline
2.  & 2. \\ \hline
3. & 3. \\ \hline
4.  & 4. \\ \hline
Body:  & Body: \\ \hline


\end{tabular}
\vspace{3mm}

\input{boxtable}
\end {multicols}
\thispagestyle{empty} 
\renewcommand{\btbinfo}[6]{
\ifx Z#6  \else \large{#1}  \fi

}
\begin{multicols}{2}
\input{boxtable}
\end {multicols}
\pagebreak
\thispagestyle{empty} 
\newcommand{\faulinfo}[5]{
#4 & #1 & #3 & #5 \\

}

\begin{tabular}[t]{|p{3cm}|p{4cm}|p{1cm}|p{6cm}    |}
\input{faultable}
\end{tabular}
\pagebreak
\setcounter{tocdepth}{1}
\tableofcontents
\end{document}
